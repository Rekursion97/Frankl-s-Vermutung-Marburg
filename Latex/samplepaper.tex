% This is samplepaper.tex, a sample chapter demonstrating the
% LLNCS macro package for Springer Computer Science proceedings;
% Version 2.21 of 2022/01/12
%
\documentclass[runningheads]{llncs}
%
\usepackage[T1]{fontenc}
% T1 fonts will be used to generate the final print and online PDFs,
% so please use T1 fonts in your manuscript whenever possible.
% Other font encondings may result in incorrect characters.
%
\usepackage{graphicx}
% Used for displaying a sample figure. If possible, figure files should
% be included in EPS format.
%
% If you use the hyperref package, please uncomment the following two lines
% to display URLs in blue roman font according to Springer's eBook style:
%\usepackage{color}
%\renewcommand\UrlFont{\color{blue}\rmfamily}
%
\begin{document}
%
\title{Frankls Vermutung}
%
%\titlerunning{Abbreviated paper title}
% If the paper title is too long for the running head, you can set
% an abbreviated paper title here
%
\author{Patrick Grieser}
%
\maketitle              % typeset the header of the contribution
%
\begin{abstract}
Frankls Vermutung besagt, dass in jeder unter Vereinigung abgeschlossener Mengenfamilie, es mindestens ein Element gibt, dass in 50\% der Mengen vorkommt.
\end{abstract}
%
%
%
\section{Unter Vereinigung abgeschlossene Mengenfamilie}
Eine Mengenfamilie $F$ ist unter Vereinigung abgeschlossen, wenn gilt $\forall M_{1},M_{2} \in F,\ M_{1}\cup M_{2} \in F$.
\subsection{Reduktion der Mengenfamilie}
Gegeben eine unter Vereinigung abgeschlossene Mengenfamilie $F$. Sei $$F_{red} = \{M \in F\ |\ \nexists M_{1},M_{2} \in F\ mit\ M_{1} \cup M_{2} = M\}$$ 
Folglich ist $F$ komplett aus $F_{red}$ reproduzierbar, da für alle $M \in F$ gilt $M = M_{1} \cup \dots \cup M_{n},\ n \leq \# F_{red}$ 
\subsubsection{Beispiel:}
$F = \{\{1, 2, 3, 4\},\{1,2\},\{1,2,3\},\{1,2,4\},\{3\},\{4\},\{3,4\}\}$. Somit entspricht $F_{red} = \{\{1,2\},\{3\},\{4\}\}$.
\section{Beweis}
Sei $F$ eine unter Vereinigung abgeschlossene Mengenfamilie und $F_{red}$ die zugehörige Reduktion von $F$. Sei $e_{max}$ das am häufigsten vorkommende Element in $F_{red}$.
 


\subsubsection{Acknowledgements} Please place your acknowledgments at
the end of the paper, preceded by an unnumbered run-in heading (i.e.
3rd-level heading).

%
% ---- Bibliography ----
%
% BibTeX users should specify bibliography style 'splncs04'.
% References will then be sorted and formatted in the correct style.
%
% \bibliographystyle{splncs04}
% \bibliography{mybibliography}
%
\begin{thebibliography}{8}
\bibitem{ref_article1}
Author, F.: Article title. Journal \textbf{2}(5), 99--110 (2016)

\bibitem{ref_lncs1}
Author, F., Author, S.: Title of a proceedings paper. In: Editor,
F., Editor, S. (eds.) CONFERENCE 2016, LNCS, vol. 9999, pp. 1--13.
Springer, Heidelberg (2016). \doi{10.10007/1234567890}

\bibitem{ref_book1}
Author, F., Author, S., Author, T.: Book title. 2nd edn. Publisher,
Location (1999)

\bibitem{ref_proc1}
Author, A.-B.: Contribution title. In: 9th International Proceedings
on Proceedings, pp. 1--2. Publisher, Location (2010)

\bibitem{ref_url1}
LNCS Homepage, \url{http://www.springer.com/lncs}. Last accessed 4
Oct 2017
\end{thebibliography}
\end{document}
